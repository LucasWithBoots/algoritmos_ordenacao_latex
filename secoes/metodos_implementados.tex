\section{Métodos Implementados}
O objetivo é ordenar vetores de números não repetidos em três formas diferentes: crescente, aleatória e decrescente, e contabilizar o número de operações de cada método para posterior comparação

Todo o ambiente de desenvolvimento está disponível para consulta e análise no \href{https://github.com/LucasWithBoots/algoritmos_ordenacao_cpp/blob/master/main.cpp}{repositório hospedado na plataforma Github}

A função $gerarVetor$ é responsável por inicializar o vetor com valores em ordem crescente, decrescente ou aleatória, dependendo do parâmetro ordem. Ela utiliza a função rand() para gerar números aleatórios e garante que não haja repetição de elementos no vetor aleatório.

\begin{minted}{c++}
void gerarVetor(int vetor[], int tamanho, string ordem) {
    if (ordem == "crescente") {
        for (int i = 0; i < tamanho; ++i) {
            vetor[i] = i + 1;
        }
    } else if (ordem == "decrescente") {
        for (int i = 0; i < tamanho; ++i) {
            vetor[i] = tamanho - i;
        }
    } else if (ordem == "aleatorio") {
        srand(time(0));
        for (int i = 0; i < tamanho; ++i) {
            vetor[i] = rand() % tamanho + 1;
            for (int j = 0; j < i; ++j) {
                if (vetor[i] == vetor[j]) {
                    --i;
                    break;
                }
            }
        }
    }
}
\end{minted}

$bubbleSort$, $insertSort$ e $selectionSort$ implementam respectivamente o algoritmo de ordenação Bubble Sort, Insertion Sort e Selection Sort. Contadores de leitura e escrita são incrementados conforme as operações são realizadas.

\begin{minted}{c++}
void bubbleSort(int arr[], int n) {...}
void insertSort(int arr[], int n) {...}
void selectionSort(int arr[], int n) {...}
\end{minted}

A função $printArray$ imprime o conteúdo do vetor. É útil para verificar o estado do vetor antes e depois da ordenação.

\begin{minted}{c++}
void printArray(int arr[], int size) {...}
\end{minted}