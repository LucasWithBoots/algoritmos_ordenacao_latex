\section{Referencial Teórico}

A ordenação de dados é um tópico amplamente estudado em ciência da computação devido à sua importância em diversas aplicações práticas e teóricas. Existem inúmeros algoritmos de ordenação, cada um com características específicas que os tornam mais ou menos adequados para determinadas situações. Nesta seção, serão discutidos alguns dos principais algoritmos de ordenação não recursivos.

\subsection{Bubble Sort}
Bubble Sort é um algoritmo que funciona comparando e trocando dois elementos se não estiver classificado e executando iterações para obter o resultado classificado\cite{analiseBubbleSort:2021}. Ele possui uma complexidade de tempo O($n^2$), o que significa que sua eficiência diminui drasticamente em listas com um número maior de elementos\cite{al2013review}.

\subsection{Insertion Sort}
Algoritmo de ordenação simples que constrói a lista final ordenando um item de cada vez. Ele possui uma complexidade de tempo O($n^2$). O Insertion Sort oferece várias vantagens: é de implementação simples e eficiente para conjuntos de dados pequenos\cite{al2013review}.

\subsection{Selection Sort}
O selection sort é um algoritmo no qual elementos sucessivos são selecionados em ordem e colocados em suas posições corretas na ordenação\cite{chand2011upgraded}. Sua complexidade de tempo é O($n^2$), não sendo considerado um algoritmo estável\cite{treinawebConheaPrincipais}.