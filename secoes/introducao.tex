\section{Introdução}
A ordenação de dados é uma operação fundamental em ciência da computação, com aplicações em uma ampla gama de campos, desde bancos de dados até algoritmos de busca e otimização. A eficiência e o desempenho dos algoritmos de ordenação têm um impacto significativo no tempo de execução de sistemas computacionais, especialmente quando lidamos com conjuntos de dados volumosos.

Este projeto tem como objetivo principal explorar e comparar diferentes métodos de ordenação não recursivos em linguagem C/C++, analisando suas eficiências em três cenários distintos: ordenação de vetores dispostos de forma crescente, aleatória e decrescente. A comparação entre os métodos será realizada em relação à utilização dos vetores, contabilizando todas as operações de leitura e escrita.

Além disso, o projeto visa fornecer uma oportunidade para compreender como relatar experimentos computacionais de maneira estruturada e informativa. O relatório resultante abordará aspectos como introdução ao problema, revisão teórica dos métodos de ordenação, descrição do material utilizado, detalhes da implementação dos métodos, análise dos resultados obtidos e conclusões tiradas a partir desses resultados.