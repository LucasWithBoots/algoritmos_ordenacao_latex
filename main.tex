%%%%%%%%%%%%%%%%%%%%%%%%%%%%%%%%%%%%%%%%%
% Diaz Essay
% LaTeX Template
% Version 2.0 (13/1/19)
%
% This template originates from:
% http://www.LaTeXTemplates.com
%
% Authors:
% Vel (vel@LaTeXTemplates.com)
% Nicolas Diaz (nsdiaz@uc.cl)
%
% License:
% CC BY-NC-SA 3.0 (http://creativecommons.org/licenses/by-nc-sa/3.0/)
%
%%%%%%%%%%%%%%%%%%%%%%%%%%%%%%%%%%%%%%%%%

%----------------------------------------------------------------------------------------
%	PACKAGES AND OTHER DOCUMENT CONFIGURATIONS
%----------------------------------------------------------------------------------------

\documentclass[11pt]{diazessay} % Font size (can be 10pt, 11pt or 12pt)

%----------------------------------------------------------------------------------------
%	TITLE SECTION
%----------------------------------------------------------------------------------------

\title{\textbf{Comparação entre os Métodos de Ordenação} \\ {\Large\itshape Relatório do desempenho do Bubble, Insertion e Selection Sort em contextos com vetores}} % Title and subtitle

\author{\textbf{Lucas Carrijo Ferrari} \\ \textit{Universidade Federal de Alfenas}} % Author and institution

\date{\today} % Date, use \date{} for no date

%----------------------------------------------------------------------------------------

\begin{document}

\maketitle % Print the title section

%----------------------------------------------------------------------------------------
%	ABSTRACT AND KEYWORDS
%----------------------------------------------------------------------------------------

%\renewcommand{\abstractname}{Summary} % Uncomment to change the name of the abstract to something else

\begin{abstract}
O objetivo deste projeto é compreender as diferenças entre métodos de ordenação não recursivos, aprender a comparar algoritmos de ordenação e relatar os experimentos realizados.
\end{abstract}

\hspace*{3.6mm}\textit{Keywords:} bubblesort, insertionsort, selectionsort, ordenação, vetores % Keywords

\vspace{30pt} % Vertical whitespace between the abstract and first section

%----------------------------------------------------------------------------------------
%	ESSAY BODY
%----------------------------------------------------------------------------------------

\section{Introdução}
A ordenação de dados é uma operação fundamental em ciência da computação, com aplicações em uma ampla gama de campos, desde bancos de dados até algoritmos de busca e otimização. A eficiência e o desempenho dos algoritmos de ordenação têm um impacto significativo no tempo de execução de sistemas computacionais, especialmente quando lidamos com conjuntos de dados volumosos.

Este projeto tem como objetivo principal explorar e comparar diferentes métodos de ordenação não recursivos em linguagem C/C++, analisando suas eficiências em três cenários distintos: ordenação de vetores dispostos de forma crescente, aleatória e decrescente. A comparação entre os métodos será realizada em relação à utilização dos vetores, contabilizando todas as operações de leitura e escrita.

Além disso, o projeto visa fornecer uma oportunidade para compreender como relatar experimentos computacionais de maneira estruturada e informativa. O relatório resultante abordará aspectos como introdução ao problema, revisão teórica dos métodos de ordenação, descrição do material utilizado, detalhes da implementação dos métodos, análise dos resultados obtidos e conclusões tiradas a partir desses resultados.
\section{Referencial Teórico}

A ordenação de dados é um tópico amplamente estudado em ciência da computação devido à sua importância em diversas aplicações práticas e teóricas. Existem inúmeros algoritmos de ordenação, cada um com características específicas que os tornam mais ou menos adequados para determinadas situações. Nesta seção, serão discutidos alguns dos principais algoritmos de ordenação não recursivos.

\subsection{Bubble Sort}
Bubble Sort é um algoritmo que funciona comparando e trocando dois elementos se não estiver classificado e executando iterações para obter o resultado classificado\cite{analiseBubbleSort:2021}. Ele possui uma complexidade de tempo O($n^2$), o que significa que sua eficiência diminui drasticamente em listas com um número maior de elementos\cite{al2013review}.

\subsection{Insertion Sort}
Algoritmo de ordenação simples que constrói a lista final ordenando um item de cada vez. Ele possui uma complexidade de tempo O($n^2$). O Insertion Sort oferece várias vantagens: é de implementação simples e eficiente para conjuntos de dados pequenos\cite{al2013review}.

\subsection{Selection Sort}
O selection sort é um algoritmo no qual elementos sucessivos são selecionados em ordem e colocados em suas posições corretas na ordenação\cite{chand2011upgraded}. Sua complexidade de tempo é O($n^2$), não sendo considerado um algoritmo estável\cite{treinawebConheaPrincipais}.
\section{Métodos Implementados}
O objetivo é ordenar vetores de números não repetidos em três formas diferentes: crescente, aleatória e decrescente, e contabilizar o número de operações de cada método para posterior comparação.

Todo o ambiente de desenvolvimento está disponível para consulta e análise no \href{https://github.com/LucasWithBoots/algoritmos_ordenacao_cpp.git}{repositório hospedado na plataforma Github}, que inclui o \href{https://github.com/LucasWithBoots/algoritmos_ordenacao_cpp/blob/master/assets/vetor.txt}{arquivo com 100.000 números gerados aleatoriamente}.

A função $gerarVetor$ é responsável por inicializar o vetor com valores em ordem crescente, decrescente ou aleatória, dependendo do parâmetro ordem. Ela utiliza a função $rand()$ para gerar números aleatórios e garante que não haja repetição de elementos no vetor aleatório.

\begin{minted}{c++}
void gerarVetor(int vetor[], int tamanho, string ordem) {
    if (ordem == "crescente") {
        for (int i = 0; i < tamanho; ++i) {
            vetor[i] = i + 1;
        }
    } else if (ordem == "decrescente") {
        for (int i = 0; i < tamanho; ++i) {
            vetor[i] = tamanho - i;
        }
    } else if (ordem == "aleatorio") {
        srand(time(0));
        for (int i = 0; i < tamanho; ++i) {
            vetor[i] = rand() % tamanho + 1;
            for (int j = 0; j < i; ++j) {
                if (vetor[i] == vetor[j]) {
                    --i;
                    break;
                }
            }
        }
    }
}
\end{minted}

$bubbleSort$, $insertSort$ e $selectionSort$ implementam respectivamente o algoritmo de ordenação Bubble Sort, Insertion Sort e Selection Sort. Contadores de leitura e escrita são incrementados conforme as operações são realizadas.

\begin{minted}{c++}
void bubbleSort(int arr[], int n) {...}
void insertSort(int arr[], int n) {...}
void selectionSort(int arr[], int n) {...}
\end{minted}

A função $printArray$ imprime o conteúdo do vetor. É útil para verificar o estado do vetor antes e depois da ordenação.

\begin{minted}{c++}
void printArray(int arr[], int size) {...}
\end{minted}
\section{Resultados Obtidos}

\vspace{.5cm}
\begin{center}
\begin{tikzpicture}
\begin{axis}[
    scale =1.5,
    title={Comparação por leitura},
    xlabel={Tamanho do Vetor},
    ylabel={Contador de Leitura},
    legend pos=north west,
    ymajorgrids=true,
    xmin=0, % Define o limite inferior do eixo x
    xmax=5000, % Define o limite superior do eixo x
    grid style=dashed,
]

\addplot[
    color=blue,
    mark=square,
    ]
    table [x=Tamanho, y=BubbleSort, col sep=tab] {leitura.dat};
    \addlegendentry{BubbleSort}
\addplot[
    color=red,
    mark=*,
    ]
    table [x=Tamanho, y=InsertionSort, col sep=tab] {leitura.dat};
    \addlegendentry{InsertionSort}
\addplot[
    color=black,
    mark=triangle,
    ]
    table [x=Tamanho, y=SelectionSort, col sep=tab] {leitura.dat};
    \addlegendentry{SelectionSort}

\end{axis}
\end{tikzpicture}

\end{center}

\begin{center}

\begin{tikzpicture}
\begin{axis}[
    scale=1.5,
    title={Comparação por escrita},
    xlabel={Tamanho do Vetor},
    ylabel={Contador de Escrita},
    legend pos=north west,
    ymajorgrids=true,
    grid style=dashed,
]
\addplot[
    color=blue,
    mark=square,
    ]
    table [x=Tamanho, y=BubbleSort, col sep=tab] {escrita.dat};
    \addlegendentry{BubbleSort}
\addplot[
    color=red,
    mark=*,
    ]
    table [x=Tamanho, y=InsertionSort, col sep=tab] {escrita.dat};
    \addlegendentry{InsertionSort}
\addplot[
    color=black,
    mark=triangle,
    ]
    table [x=Tamanho, y=SelectionSort, col sep=tab] {escrita.dat};
    \addlegendentry{SelectionSort}

\end{axis}
\end{tikzpicture}


\end{center}


\section{Conclusão}

Após a análise dos dados dispostos, podemos chegar a algumas conclusões acerca dos algoritmos de ordenação:

\subsection{Bubble Sort}

O Bubble Sort, conhecido por sua simplicidade, apresentou o pior desempenho, especialmente para vetores aleatórios e decrescentes. Em vetores aleatórios, o número de operações cresce quadraticamente, chegando a 19.998.000.000 para um vetor de tamanho 100000. Para vetores crescentes, o desempenho é linear, necessitando de apenas 199.998 operações para o mesmo tamanho, refletindo seu melhor caso.

\subsection{Insertion Sort}

O Insertion Sort teve desempenho razoável em vetores aleatórios, melhor que o Bubble Sort, mas ainda com crescimento rápido no número de operações (74.956.944 para um vetor de tamanho 10000). Em vetores crescentes, seu desempenho foi excelente, com operações lineares (199.998 para um vetor de tamanho 100000). No entanto, apresentou o pior desempenho para vetores decrescentes, com 15.000.049.998 operações para um vetor de tamanho 100000.

\subsection{Selection Sort}

O Selection Sort, embora mais eficiente que o Bubble Sort, não teve um desempenho tão bom quanto o Insertion Sort. Para vetores aleatórios, o número de operações chegou a 50.104.154 para um vetor de tamanho 10000. Em vetores crescentes, manteve um comportamento quadrático (50.004.999 operações para um vetor de tamanho 100000). Para vetores decrescentes, teve desempenho intermediário com 75.014.999 operações para um vetor de tamanho 100000.

%----------------------------------------------------------------------------------------
%	BIBLIOGRAPHY
%----------------------------------------------------------------------------------------

\bibliographystyle{unsrt}

\bibliography{bibliografia.bib}

%----------------------------------------------------------------------------------------

\end{document}
